\documentclass[journal,12pt,twocolumn]{IEEEtran}

\usepackage{setspace}
\usepackage{gensymb}
\singlespacing
\usepackage[cmex10]{amsmath}

\usepackage{amsthm}

\usepackage{mathrsfs}
\usepackage{txfonts}
\usepackage{stfloats}
\usepackage{bm}
\usepackage{cite}
\usepackage{cases}
\usepackage{subfig}

\usepackage{longtable}
\usepackage{multirow}

\usepackage{enumitem}
\usepackage{mathtools}
\usepackage{steinmetz}
\usepackage{tikz}
\usepackage{circuitikz}
\usepackage{verbatim}
\usepackage{tfrupee}
\usepackage[breaklinks=true]{hyperref}
\usepackage{graphicx}
\usepackage{tkz-euclide}

\usetikzlibrary{calc,math}
\usepackage{listings}
    \usepackage{color}                                            %%
    \usepackage{array}                                            %%
    \usepackage{longtable}                                        %%
    \usepackage{calc}                                             %%
    \usepackage{multirow}                                         %%
    \usepackage{hhline}                                           %%
    \usepackage{ifthen}                                           %%
    \usepackage{lscape}     
\usepackage{multicol}
\usepackage{chngcntr}

\DeclareMathOperator*{\Res}{Res}

\renewcommand\thesection{\arabic{section}}
\renewcommand\thesubsection{\thesection.\arabic{subsection}}
\renewcommand\thesubsubsection{\thesubsection.\arabic{subsubsection}}

\renewcommand\thesectiondis{\arabic{section}}
\renewcommand\thesubsectiondis{\thesectiondis.\arabic{subsection}}
\renewcommand\thesubsubsectiondis{\thesubsectiondis.\arabic{subsubsection}}


\hyphenation{op-tical net-works semi-conduc-tor}
\def\inputGnumericTable{}                                 %%

\lstset{
%language=C,
frame=single, 
breaklines=true,
columns=fullflexible
}
\begin{document}


\newtheorem{theorem}{Theorem}[section]
\newtheorem{problem}{Problem}
\newtheorem{proposition}{Proposition}[section]
\newtheorem{lemma}{Lemma}[section]
\newtheorem{corollary}[theorem]{Corollary}
\newtheorem{example}{Example}[section]
\newtheorem{definition}[problem]{Definition}

\newcommand{\BEQA}{\begin{eqnarray}}
\newcommand{\EEQA}{\end{eqnarray}}
\newcommand{\define}{\stackrel{\triangle}{=}}
\bibliographystyle{IEEEtran}
\raggedbottom
\setlength{\parindent}{0pt}
\providecommand{\mbf}{\mathbf}
\providecommand{\pr}[1]{\ensuremath{\Pr\left(#1\right)}}
\providecommand{\qfunc}[1]{\ensuremath{Q\left(#1\right)}}
\providecommand{\sbrak}[1]{\ensuremath{{}\left[#1\right]}}
\providecommand{\lsbrak}[1]{\ensuremath{{}\left[#1\right.}}
\providecommand{\rsbrak}[1]{\ensuremath{{}\left.#1\right]}}
\providecommand{\brak}[1]{\ensuremath{\left(#1\right)}}
\providecommand{\lbrak}[1]{\ensuremath{\left(#1\right.}}
\providecommand{\rbrak}[1]{\ensuremath{\left.#1\right)}}
\providecommand{\cbrak}[1]{\ensuremath{\left\{#1\right\}}}
\providecommand{\lcbrak}[1]{\ensuremath{\left\{#1\right.}}
\providecommand{\rcbrak}[1]{\ensuremath{\left.#1\right\}}}
\theoremstyle{remark}
\newtheorem{rem}{Remark}
\newcommand{\sgn}{\mathop{\mathrm{sgn}}}
\providecommand{\abs}[1]{\left\vert#1\right\vert}
\providecommand{\res}[1]{\Res\displaylimits_{#1}} 
\providecommand{\norm}[1]{\left\lVert#1\right\rVert}
%\providecommand{\norm}[1]{\lVert#1\rVert}
\providecommand{\mtx}[1]{\mathbf{#1}}
\providecommand{\mean}[1]{E\left[ #1 \right]}
\providecommand{\fourier}{\overset{\mathcal{F}}{ \rightleftharpoons}}
%\providecommand{\hilbert}{\overset{\mathcal{H}}{ \rightleftharpoons}}
\providecommand{\system}{\overset{\mathcal{H}}{ \longleftrightarrow}}
	%\newcommand{\solution}[2]{\textbf{Solution:}{#1}}
\newcommand{\solution}{\noindent \textbf{Solution: }}
\newcommand{\cosec}{\,\text{cosec}\,}
\providecommand{\dec}[2]{\ensuremath{\overset{#1}{\underset{#2}{\gtrless}}}}
\newcommand{\myvec}[1]{\ensuremath{\begin{pmatrix}#1\end{pmatrix}}}
\newcommand{\mydet}[1]{\ensuremath{}}}
\numberwithin{equation}{subsection}

\makeatletter
\@addtoreset{figure}{problem}
\makeatother
\let\StandardTheFigure\thefigure
\let\vec\mathbf

\renewcommand{\thefigure}{\theproblem}

\def\putbox#1#2#3{\makebox[0in][l]{\makebox[#1][l]{}\raisebox{\baselineskip}[0in][0in]{\raisebox{#2}[0in][0in]{#3}}}}
     \def\rightbox#1{\makebox[0in][r]{#1}}
     \def\centbox#1{\makebox[0in]{#1}}
     \def\topbox#1{\raisebox{-\baselineskip}[0in][0in]{#1}}
     \def\midbox#1{\raisebox{-0.5\baselineskip}[0in][0in]{#1}}
\vspace{3cm}
\title{Gate Assignment 1}
\author{Tanmay Goyal - AI20BTECH11021}
\maketitle
\newpage
\bigskip
\renewcommand{\thefigure}{\theenumi}
\renewcommand{\thetable}{\theenumi}
Download all python codes from 
\begin{lstlisting}
https://github.com/tanmaygoyal258/EE3900-Linear-Systems-and-Signal-processing/blob/main/GateAssignment1/code.py
\end{lstlisting}
Download all latex codes from 
\begin{lstlisting}
https://github.com/tanmaygoyal258/EE3900-Linear-Systems-and-Signal-processing/blob/main/GateAssignment1/main.tex
\end{lstlisting}
\section{Problem}
(EC 2017- Q.7) The input $x(t)$ and output $y(t)$ of a continous time signal are related as:
\begin{align}
    y(t) = \int_{t-T}^tx(u)\,du
\end{align}
The system is:
\begin{enumerate}
    \item Linear and Time-variant
    \item Linear and Time-invariant
    \item Non-Linear and Time-variant
    \item Non-Linear and Time-invariant
\end{enumerate}
\section{Solution}
\begin{definition}
We say that a system is\textbf{ linear} if and only if it follows the Principle of Superposition, i.e Law of Additivity and Law of Homogeneity.
\label{L}
\end{definition}
\begin{definition}
A system is said to be \textbf{time invariant} if the output signal does not depend on the absolute time, i.e a time delay on the input signal directly equates to the delay in the output signal.
\label{T}
\end{definition}
\begin{lemma}
The system relating the input signal $x(t$) and output signal $y(t)$, given by 
\begin{align}
     y(t) = \int_{t-T}^tx(u)\,du
\end{align}
is linear and time invariant in nature.
\end{lemma}
\begin{proof}
\begin{enumerate}
\item \textbf{\underline{Linearity and Time invariance}}\\
From \eqref{L}, we can say the system is linear if it follows both the laws of Additivity and Homogeneity.\\
\underline{Law of Additivity:}\\
Let the two input signals be $x_1(t)$ and $x_2(t)$, and their corresponding output signals be $y_1(t)$ and $y_2(t)$, then:
\begin{align}
    y_1(t) = \int_{t-T}^tx_1(u)\,du\\
    y_2(t) = \int_{t-T}^tx_2(u)\,du\\
    y_1(t) + y_2(t) = \int_{t-T}^t[x_1(u) + x_2(u)]\,du
    \label{1}
\end{align}
Now, consider the input signal of $x_1(t) + x_2(t)$, then the corresponding output signal is given by $y'(t)$:
\begin{align}
    y'(t) = \int_{t-T}^t[x_1(u) + x_2(u)]\,du
    \label{2}
\end{align}
Clearly, from \eqref{1} and \eqref{2}:
\begin{align}
    y'(t) = y_1(t) + y_2(t)
\end{align}
Thus, the Law of Additivity holds.\\

\underline{Law of Homogeneity: }\\
Consider an input signal $kx(t)$, where $k$ is any constant. Let the corresponding output be given by $y'(t)$, then:
\begin{align}
    y'(t) = \int_{t-T}^t kx(u)\,du\\
    = k\int_{t-T}^t x(u)\,du\\
     = ky(t)
     \label{3}
\end{align}
Clearly, from \eqref{3},
\begin{align}
    y'(t) = ky(t)
\end{align}
Thus, the Law of Homogeneity holds.\\

Since both the Laws hold, the system satisfies the Principle of Superposition, and is thus, a \textbf{linear system}.\\

From \eqref{T} , to check for time-invariance, we would introduce a delay of $t_0$ in the output and input signals.\\
Delay in output signal:
\begin{align}
    y(t-t_0) = \int_{t-t_0-T}^{t-t_0} x(u)\,du
    \label{4}
\end{align}
Now, we consider an input signal with a delay of $t_0$, given by $x(t-t_0)$, and let the corresponding output signal be given by $y'(t)$, then:
\begin{align}
    y'(t) = \int_{t-T}^{t} x(u-t_0)\,du
\end{align}
Substituting $a = u-t_0$:
\begin{align}
    y'(t) = \int_{t-t_0-T}^{t-t_0} x(a)\,da
    \label{5}
\end{align}
Clearly, from \eqref{4} and \eqref{5}:
\begin{align}
    y'(t) = y(t-t_0)
\end{align}
Thus, the system is \textbf{time-invariant}.\\
The correct option is \textbf{2) Linear and Time-invariant}\\


\begin{figure}[!ht]
\centering
 \includegraphics[width=\columnwidth]{graphs/input_signals.png}
 \caption{$x_1(t) = \sin{t}$ and $x_2(t) = t$}
 \end{figure}
\begin{figure}[!ht]
\centering
 \includegraphics[width=\columnwidth]{graphs/output_signals.png}
 \caption{$y_1(t)$ and  $y_2(t)$}
 \end{figure}
 \begin{figure}[!ht]
\centering
 \includegraphics[width=\columnwidth]{graphs/law_of_additivity.png}
 \caption{Law of Additivity}
 \end{figure}
 \begin{figure}[!ht]
\centering
 \includegraphics[width=\columnwidth]{graphs/law_of_homogeneity.png}

 \caption{Law of Homogeneity}
 \end{figure}
  \begin{figure}[!ht]
\centering
 \includegraphics[width=\columnwidth]{graphs/time_invariance.png}
 \caption{Time invariance}
 \end{figure}

\item \textbf{\underline{Calculating impulse response of LTI system}}\\
Since the given system is an LTI system, it would possess an impulse response $h(t)$, which is the output of the system when the input signal is the Impulse function, given by $\delta(t)$. Thus,
\begin{align}
    h(t) = \int_{t-T}^{t} \delta(u)du
\end{align}\\
\\
The Impulse function can be loosely defined as:
\begin{align}
    \delta(t) = 
    \begin{cases}
\infty & t = 0\\
0 & otherwise
\end{cases}
and \int_{-\infty}^\infty \delta(t)dt  = 1
\end{align}
Since the Impulse function is zero everywhere aside from $t = 0$ , the non-zero value of integration is a result of $\delta(0)$. Thus, we can say $h(t)$ will be non-zero only if the limits of integration would include $t=0$, i.e:
\begin{align}
    h(t) = 
    \begin{cases}
    \int_{t-T}^{t} \delta(u)du & t-T<0 ; t>0\\
    0 & otherwise
    \end{cases}
    \end{align}
    \begin{align}
h(t) = 
    \begin{cases}
    1 & 0<t<T\\
    0 & otherwise
    \end{cases}
    \label{H}
\end{align}
\item \textbf{\underline{Expressing the impulse function in terms of $u(t)$}} \\
The unit step signal, $u(t)$, is given by:
\begin{align}
    u(t) = 
    \begin{cases}
    1 & t\geq0\\
    0 & otherwise
    \end{cases}
    \label{u(t)}
\end{align}
On time-shifting $u(t)$ by T, we get:
\begin{align}
     u(t - T) = 
    \begin{cases}
    1 & t-T\geq 0\\
    0 & otherwise
    \end{cases}
    = 
    \begin{cases}
    1 & t\geq T\\
    0 & otherwise
    \end{cases}
    \label{u(t-T)}
\end{align}
On subtracting \eqref{u(t)} and \eqref{u(t-T)}, we get our impulse response $h(t)$ in terms of the unit step signal:
\begin{align}
    h(t) = u(t) - u(t-T)
\end{align}
\item \textbf{\underline{Expressing the impulse function in terms of $rect(t)$}}\\
The unit rectangular signal, $rect(t)$ is given by:
\begin{align}
    rect(t) = 
    \begin{cases}
    1 & \frac{-1}{2} \leq t \leq \frac{1}{2} \\
    0 & otherwise
    \end{cases}
    \label{rect}
\end{align}
We can obtain the impulse response $h(t)$ in terms of $rect(t)$ using time scaling and shifting as follows:
\begin{align}
    rect\brak{\frac{t}{\tau}} = 
    \begin{cases}
    1 & \frac{-1}{2} \leq \frac{t}{\tau} \leq  \frac{1}{2}\\
    0 & otherwise
    \end{cases}
     = 
     \begin{cases}
     1 & \frac{-\tau}{2} \leq t \leq  \frac{\tau}{2}\\
    0 & otherwise
     \end{cases}
    \end{align}
    Subsituting $\tau = T$:
    \begin{align}
    rect\brak{\frac{t}{T}} = 
    \begin{cases}
    1 & \frac{-T}{2} \leq t \leq \frac{T}{2}\\
    0 & otherwise
    \end{cases}
    \end{align}
    Now, we want to right-shift the signal by $\frac{T}{2}$:
    \begin{align}
    rect\brak{\frac{1}{T}\brak{t - \frac{T}{2}}} = 
    \begin{cases}
    1 &  0 \leq t \leq T\\
    0 & otherwise
    \end{cases}
     = h(t)
\end{align}
Since the time shifting is to be performed on the variable $t$ and not $\frac{t}{T}$\\

\item 
\textbf{\underline{Calculating the Fourier Transform of $h(t)$}}\\
Let the Fourier Transform of $h(t)$ be given by $H(f)$ and of the rectangular signal, $rect(t)$ be given by $Y(f)$.
\begin{align}
    h(t) \fourier H(f)\\
    rect(t) \fourier Y(f)
    \end{align}
    Then,
    \begin{align}
    Y(f) = \int_{-\infty}^\infty rect(t)e^{-j2\pi f t}\,dt
    \label{Fourier}
\end{align}
From \eqref{rect}, we can write \eqref{Fourier} as:
\small{
\begin{align}
   Y(f) = \int_{-\infty}^\frac{-1}{2} 0\,dt + \int_{\frac{-1}{2}}^\frac{1}{2} e^{-j2\pi ft}\,dt + \int_\frac{1}{2}^\infty 0\,dt\\
    = \frac{e^{j\pi f} - e^{-j \pi f}}{j2\pi f}\\
     = \frac{2j\sin{\pi f}}{j2\pi f}\\
      = \frac{\sin (\pi f)}{\pi f}\\
       = sinc(f)
\end{align}
}
where $sinc(t)$, the sampling function is defined as:
\begin{align}
    sinc(t) = 
    \begin{cases}
    1 & t = 0\\
    \frac{\sin(\pi t)}{\pi t} & otherwise
    \end{cases}
\end{align}
Let the Fourier Transform of a signal $x(t)$ be $X(f)$.
\begin{align}
    x(t) \fourier X(f)
\end{align}
When the signal $x(t)$ is time shifted by $t_0$, the resultant Fourier Transform is given by:
\begin{align}
    x(t \pm t_0) \fourier X(f)e^{\pm j2\pi ft_0}
    \label{shift}
\end{align}
And when the signal $x(t)$ is time scaled by $\alpha$, the resulting Fourier Transform is given by:
\begin{align}
    x(\alpha t) \fourier \frac{1}{\abs{\alpha}}X\brak{\frac{f}{\alpha}}
    \label{scale}
\end{align}
Since we have already derived the Fourier Transform of $rect(t)$, we would use the properties mentioned above to find the Fourier Transform of $h(t)$:
\begin{align}
    rect(t) \fourier sinc( f)
\end{align}
Using \eqref{shift}:
\begin{align}
    rect\brak{t - \frac{T}{2}} \fourier sinc( f)e^{-j(2 \pi f)\frac{T}{2}}\\
    rect\brak{t - \frac{T}{2}} \fourier sinc( f)e^{-j \pi f T}
\end{align}
Using \eqref{scale},
\begin{align}
    rect\brak{\frac{1}{T}\brak{t - \frac{T}{2}}} \fourier \frac{1}{\frac{1}{\abs{T}}}sinc\brak{\frac{ f }{T}}e^{\frac{-j \pi f T}{T}}\\
    h(t) \fourier Tsinc\brak{\frac{ f }{T}}e^{-j\pi f}\\
    \therefore H(f)  = Tsinc\brak{\frac{ f }{T}}e^{-j\pi f}
\end{align}\\


\item \textbf{\underline{An example}}\\
Consider an input signal of $x(t) = \cos{2\pi f_0t}$. The Fourier Transform of $x(t)$ is given by:
\begin{align}
    x(t) = \cos{2\pi f_0t} \fourier \frac{1}{2}\sbrak{\delta(f - f_0) + \delta(f + f_0)}
\end{align}
using the fact that 
\begin{align}
    \cos{2\pi f_0 t} = \frac{e^{j2\pi f_0 t} + e^{-j2\pi f_0 t}}{2}
\end{align} and the Fourier Transform of $e^{\pm j2\pi f_0 t}$ is given by:
\begin{align}
    e^{\pm j2\pi  f_0 t} \fourier \delta(f \mp f_0)
    \label{FT}
\end{align}
The output signal will be given by:
\begin{align}
    y(t) = \int_{t-T}^t \cos{2\pi f_0u}\,du\\
     = \frac{1}{2\pi f_0}\sbrak{\sin{2\pi f_0t}- \sin2\pi f_0({t-T})}\\
      = \frac{\sin\pi f_0 T}{\pi f_0} \sbrak{\cos2\pi f_0 \brak{t - \frac{T}{2}}}\\
      = Tsinc(f_0 T)\cos2\pi f_0\brak{t - \frac{T}{2}}
      \label{output}
\end{align}
The Fourier transform of $\cos2\pi f_0\brak{t - \frac{T}{2}}$ can be obtained using \eqref{scale} and \eqref{shift} as follows:
\small{
\begin{align}
    \cos t = \frac{1}{2}\sbrak{e^{jt} + e^{-jt}}\\
    \cos t \fourier \frac{1}{2}\sbrak{\delta\brak{f - \frac{1}{2\pi}} + \delta\brak{f + \frac{1}{2\pi}}}\\
    \cos\brak{t - \frac{T}{2}} \fourier \frac{e^{j\pi fT}}{2}\sbrak{\delta\brak{f - \frac{1}{2\pi}} + \delta\brak{f + \frac{1}{2\pi}}} \\
    \cos2\pi f_0\brak{t - \frac{T}{2}} \fourier \frac{e^{j\pi \frac{f}{2\pi f_0}T}}{2\pi f_0}\frac{\delta(\frac{f}{2\pi f_0} - \frac{1}{2\pi}) + \delta(\frac{f}{2\pi f_0} + \frac{1}{2\pi})}{2} \\
    \cos 2\pi f_0\brak{t - \frac{T}{2}} \fourier\frac{e^{j\pi \frac{f}{2f_0}T}}{4\pi f_0}\brak{\delta\brak{\frac{f - f_0}{2\pi f_0}} + \delta\brak{\frac{f + f_0}{2\pi f_0}}}
\end{align}
}
Therefore, the Fourier Transform of the output signal $y(t)$ from \eqref{output} is given by:
 \begin{align}
     y(t) \fourier \frac{Tsinc( f_0 T)}{4\pi f_0}e^{j\pi \frac{f}{2f_0}T}\brak{\delta\brak{\frac{f - f_0}{2\pi f_0}} + \delta\brak{\frac{f + f_0}{2\pi f_0}}}\\
     y(t) \fourier ke^{j\pi \frac{f}{2f_0}T}\brak{\delta\brak{\frac{f - f_0}{2\pi f_0}} + \delta\brak{\frac{f + f_0}{2\pi f_0}}}
 \end{align}
 where $k = \frac{Tsinc( f_0T)}{4\pi f_0}$. Substituting $2\pi f_0 = 1$ and $T = 1$:
 \begin{align}
     y(t) \fourier ke^{j\pi^2f}\brak{\delta\brak{f - \frac{1}{2\pi}} + \delta\brak{f + \frac{1}{2\pi}}}\\
     y(t) \fourier ke^{j\frac{\pi}{2}}\delta\brak{f - \frac{1}{2\pi}} +ke^{j\frac{-\pi}{2}}\delta\brak{f + \frac{1}{2\pi}}
 \end{align}
 using the multiplication property of the Delta function:
 \begin{align}
     x(t)\delta(t - t_1) = x(t_1)\delta(t - t_1)
 \end{align}
 Since , $e^{j\frac{\pi}{2}} = j$ and $e^{-j\frac{\pi}{2}} = -j$, we finally get:
 \begin{align}
 y(t) \fourier kj\sbrak{\delta\brak{f - \frac{1}{2\pi}} - \delta\brak{f + \frac{1}{2\pi}}}
 \end{align}
 Clearly, the Fourier transform of $y(t)$ can be manipulated to represent a  sinusoidal wave, which is given by:
 \begin{align}
     sin(t) \fourier \frac{-j}{2}\sbrak{\delta\brak{f-\frac{1}{2\pi}} - \delta\brak{f + \frac{1}{2\pi}}}
 \end{align}
The attenuation happens for the same values of $f$, as depicted in the graphs of the Fourier Transforms given below.
\begin{figure}[!ht]
\centering
 \includegraphics[width=\columnwidth]{graphs/fourier_impulse_response.png}
 \caption{Fourier Transform of Impulse response $h(t)$}
 \end{figure}
 \begin{figure}[!ht]
\centering
 \includegraphics[width=\columnwidth]{graphs/fourier_impulse_response_zoomed.png}
 \caption{Fourier Transform of Impulse response $h(t)$ zoomed in}
 \end{figure}

\end{enumerate}
\end{proof}
\end{document}
