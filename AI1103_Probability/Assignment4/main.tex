\documentclass[journal,12pt,twocolumn]{IEEEtran}

\usepackage{setspace}
\usepackage{gensymb}
\singlespacing
\usepackage[cmex10]{amsmath}

\usepackage{amsthm}

\usepackage{mathrsfs}
\usepackage{txfonts}
\usepackage{stfloats}
\usepackage{bm}
\usepackage{cite}
\usepackage{cases}
\usepackage{subfig}

\usepackage{longtable}
\usepackage{multirow}

\usepackage{enumitem}
\usepackage{mathtools}
\usepackage{steinmetz}
\usepackage{tikz}
\usepackage{circuitikz}
\usepackage{verbatim}
\usepackage{tfrupee}
\usepackage[breaklinks=true]{hyperref}
\usepackage{graphicx}
\usepackage{tkz-euclide}

\usetikzlibrary{calc,math}
\usepackage{listings}
    \usepackage{color}                                            %%
    \usepackage{array}                                            %%
    \usepackage{longtable}                                        %%
    \usepackage{calc}                                             %%
    \usepackage{multirow}                                         %%
    \usepackage{hhline}                                           %%
    \usepackage{ifthen}                                           %%
    \usepackage{lscape}     
\usepackage{multicol}
\usepackage{chngcntr}

\DeclareMathOperator*{\Res}{Res}

\renewcommand\thesection{\arabic{section}}
\renewcommand\thesubsection{\thesection.\arabic{subsection}}
\renewcommand\thesubsubsection{\thesubsection.\arabic{subsubsection}}

\renewcommand\thesectiondis{\arabic{section}}
\renewcommand\thesubsectiondis{\thesectiondis.\arabic{subsection}}
\renewcommand\thesubsubsectiondis{\thesubsectiondis.\arabic{subsubsection}}


\hyphenation{op-tical net-works semi-conduc-tor}
\def\inputGnumericTable{}                                 %%

\lstset{
%language=C,
frame=single, 
breaklines=true,
columns=fullflexible
}
\begin{document}


\newtheorem{theorem}{Theorem}[section]
\newtheorem{problem}{Problem}
\newtheorem{proposition}{Proposition}[section]
\newtheorem{lemma}{Lemma}[section]
\newtheorem{corollary}[theorem]{Corollary}
\newtheorem{example}{Example}[section]
\newtheorem{definition}[problem]{Definition}

\newcommand{\BEQA}{\begin{eqnarray}}
\newcommand{\EEQA}{\end{eqnarray}}
\newcommand{\define}{\stackrel{\triangle}{=}}
\bibliographystyle{IEEEtran}
\raggedbottom
\setlength{\parindent}{0pt}
\providecommand{\mbf}{\mathbf}
\providecommand{\pr}[1]{\ensuremath{\Pr\left(#1\right)}}
\providecommand{\qfunc}[1]{\ensuremath{Q\left(#1\right)}}
\providecommand{\sbrak}[1]{\ensuremath{{}\left[#1\right]}}
\providecommand{\lsbrak}[1]{\ensuremath{{}\left[#1\right.}}
\providecommand{\rsbrak}[1]{\ensuremath{{}\left.#1\right]}}
\providecommand{\brak}[1]{\ensuremath{\left(#1\right)}}
\providecommand{\lbrak}[1]{\ensuremath{\left(#1\right.}}
\providecommand{\rbrak}[1]{\ensuremath{\left.#1\right)}}
\providecommand{\cbrak}[1]{\ensuremath{\left\{#1\right\}}}
\providecommand{\lcbrak}[1]{\ensuremath{\left\{#1\right.}}
\providecommand{\rcbrak}[1]{\ensuremath{\left.#1\right\}}}
\theoremstyle{remark}
\newtheorem{rem}{Remark}
\newcommand{\sgn}{\mathop{\mathrm{sgn}}}
\providecommand{\abs}[1]{\left\vert#1\right\vert}
\providecommand{\res}[1]{\Res\displaylimits_{#1}} 
\providecommand{\norm}[1]{\left\lVert#1\right\rVert}
%\providecommand{\norm}[1]{\lVert#1\rVert}
\providecommand{\mtx}[1]{\mathbf{#1}}
\providecommand{\mean}[1]{E\left[ #1 \right]}
\providecommand{\fourier}{\overset{\mathcal{F}}{ \rightleftharpoons}}
%\providecommand{\hilbert}{\overset{\mathcal{H}}{ \rightleftharpoons}}
\providecommand{\system}{\overset{\mathcal{H}}{ \longleftrightarrow}}
	%\newcommand{\solution}[2]{\textbf{Solution:}{#1}}
\newcommand{\solution}{\noindent \textbf{Solution: }}
\newcommand{\cosec}{\,\text{cosec}\,}
\providecommand{\dec}[2]{\ensuremath{\overset{#1}{\underset{#2}{\gtrless}}}}
\newcommand{\myvec}[1]{\ensuremath{\begin{pmatrix}#1\end{pmatrix}}}
\newcommand{\mydet}[1]{\ensuremath{}}}
\numberwithin{equation}{subsection}

\makeatletter
\@addtoreset{figure}{problem}
\makeatother
\let\StandardTheFigure\thefigure
\let\vec\mathbf

\renewcommand{\thefigure}{\theproblem}

\def\putbox#1#2#3{\makebox[0in][l]{\makebox[#1][l]{}\raisebox{\baselineskip}[0in][0in]{\raisebox{#2}[0in][0in]{#3}}}}
     \def\rightbox#1{\makebox[0in][r]{#1}}
     \def\centbox#1{\makebox[0in]{#1}}
     \def\topbox#1{\raisebox{-\baselineskip}[0in][0in]{#1}}
     \def\midbox#1{\raisebox{-0.5\baselineskip}[0in][0in]{#1}}
\vspace{3cm}
\title{Assignment 4}
\author{Tanmay Goyal - AI20BTECH11021}
\maketitle
\newpage
\bigskip
\renewcommand{\thefigure}{\theenumi}
\renewcommand{\thetable}{\theenumi}
Download all python codes from 
\begin{lstlisting}
https://github.com/tanmaygoyal258/AI1103---Probability/blob/main/Assignment4/code.py
\end{lstlisting}
%
and latex-tikz codes from 
%
\begin{lstlisting}
https://github.com/tanmaygoyal258/AI1103---Probability/blob/main/Assignment4/main.tex
\end{lstlisting}
\section{Problem}
 The number $N$ of persons getting injured in a bomb blast at a busy market place is a random variable having a Poisson Distribution with parameter $\lambda(\geq 1)$.
 A person injured in the explosion may either suffer a minor injury requiring first aid or suffer a major injury requiring hospitalisation. Let the number of persons with minor injury be $N_1$ and the conditional distribution of $N_1$ given N is
 \begin{align}
     \pr{N_1 = i \vert N} = \frac{1}{N}
     \label{condition}
 \end{align}
 Find the expected number of persons requiring hospitalisation.
 
\section{Solution}
We know,
\begin{align}
    \pr{A|B} = \frac{\pr{A\cap B}}{\pr{B}}
\end{align}
Also, for a Poisson Distribution:
\begin{align}
    \pr{N = x} = \frac{e^{-\lambda}\lambda^x}{x!}
    \label{poisson}
\end{align}
where $\lambda$ is the parameter\\
Let $N_2$ be the number of persons hospitalised.\\
Let $N = a$, and $N_1 = i(i\leq a)$, then, $N_2 = a-i$\\
Then, from \eqref{condition} and \eqref{poisson}:
\begin{align}
    \pr{N_2 = a-i} =\pr{N_1 = i}\\
    =\pr{N_1 = i|N = a}\pr{N = a}\\
     = \frac{1}{a}\frac{e^{-\lambda}\lambda^a}{a!}
\end{align}
Thus,
\begin{align}
    E(N_2) = \sum_{a = 0}^{\infty} \sum_{i = 0 }^{a} (a-i) \times \frac{1}{a}\frac{e^{-\lambda}\lambda^a}{a!}\\
     = \sum_{a=0}^{\infty}\frac{e^{-\lambda}\lambda^a}{a!} \sum_{i = 0 }^{a} \frac{a-i}{a}\\
    = \sum_{a=0}^{\infty}\frac{e^{-\lambda}\lambda^a}{a!} \brak{a - \frac{(a+1)}{2}}\\
     = \sum_{a=0}^{\infty}\frac{e^{-\lambda}\lambda^a}{a!}\frac{a-1}{2}\\
      = \frac{e^{-\lambda}}{2}\sbrak{\sum_{a=0}^{\infty}\frac{a\lambda^a}{a!} - \sum_{a=0}^{\infty}\frac{\lambda^a}{a!}}\\
       = \frac{e^{-\lambda}}{2}\sbrak{\lambda\sum_{a=1}^{\infty} \frac{\lambda^{a-1}}{(a-1)!} - \sum_{a=0}^{\infty}\frac{\lambda^a}{a!}}\\
        = \frac{e^{-\lambda}}{2}\sbrak{\lambda e^\lambda - e^\lambda}\\
         = \frac{\lambda - 1}{2}
\end{align}
\end{document}
